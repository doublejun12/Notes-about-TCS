\documentclass{article}
\usepackage[T1]{fontenc}
\usepackage{hyperref, amssymb, amsmath, graphicx, subfigure}

\setlength{\oddsidemargin}{.25in}
\setlength{\evensidemargin}{.25in}
\setlength{\textwidth}{6in}
\setlength{\topmargin}{-0.4in}
\setlength{\textheight}{8.5in}

\setlength{\parindent}{0in}
\setlength{\parskip}{8pt}

\newcommand{\heading}[6]{
  \renewcommand{\thepage}{#1-\arabic{page}}
  \noindent
  \begin{center}
  \framebox{
    \vbox{
      \hbox to 5.78in { \textbf{#2} \hfill #3 }
      \vspace{4mm}
      \hbox to 5.78in { {\Large \hfill #6  \hfill} }
      \vspace{2mm}
      \hbox to 5.78in { \textit{Instructor: #4 \hfill #5} }
    }
  }
  \end{center}
  \vspace*{4mm}
}

\newtheorem{theorem}{Theorem}
\newtheorem{definition}[theorem]{Definition}
\newtheorem{remark}[theorem]{Remark}
\newtheorem{lemma}[theorem]{Lemma}
\newtheorem{corollary}[theorem]{Corollary}
\newtheorem{proposition}[theorem]{Proposition}
\newtheorem{claim}[theorem]{Claim}
\newtheorem{observation}[theorem]{Observation}
\newtheorem{fact}[theorem]{Fact}
\newtheorem{assumption}[theorem]{Assumption}

\newenvironment{proof}{\noindent{\bf Proof:} \hspace*{1mm}}{
	\hspace*{\fill} $\Box$ }
\newenvironment{proof_of}[1]{\noindent {\bf Proof of #1:}
	\hspace*{1mm}}{\hspace*{\fill} $\Box$ }
\newenvironment{proof_claim}{\begin{quotation} \noindent}{
	\hspace*{\fill} $\diamond$ \end{quotation}}
	
\newcommand{\lecture}[4]{\heading{#1}{Spectral Graph Theory}{#2}{Ryan O'Donnel}{Scribe: #4}{#3}}



%%%%%%%%%%%%%%%%%%%%%%%%%%%%%%%%%%%%%%%%%%%%%%%%%%%%%%%%%%%%%%%%%%%%%%%%%%%%%%%
% PLEASE MODIFY THESE FIELDS AS APPROPRIATE:
\newcommand{\lecturenum}{1} % lecture number
\newcommand{\lecturedate}{\today} % date of lecture (e.g., 'March 20, 2010')
\newcommand{\lecturetitle}{The Quadratic Form and Standard Random Walk} % lecture title
\newcommand{\scribename}{Shuangjun Zhang} % full name of scribe
% PUT HERE ANY PACKAGES, MACROS, etc., ADDED BY YOU
%
\usepackage{graphicx}
\usepackage{subfigure}
%%%%%%%%%%%%%%%%%%%%%%%%%%%%%%%%%%%%%%%%%%%%%%%%%%%%%%%%%%%%%%%%%%%%%%%%%%%%%%%


%%%%%%%%%%%%%%%%%%%%%%%%%%%%%%%%%%%%%%%%%%%%%%%%%%%%%%%%%%%%%%%%%%%%%%%%%%%%%%%
\begin{document}
\lecture{\lecturenum}{\lecturedate}{\lecturetitle}{\scribename}

\section{The Labelling Function}

We consider the undirected graph $G=\left(V, E\right) $ and it satisfies the following properties:
\begin{itemize}
  \item The graph is finite.
  \item Multiple parallel edges and self-loops are allowed.
  \item vertices of degree $0$ are not allowed.
\end{itemize}
For simplicity, maybe we assume $G$ is regular.

\par We can label the vertice set $V$ by real numbers: 
$$f: V \rightarrow \mathbb{R} \equiv\left[\begin{array}{c}
  f\left(v_{1}\right) \\
  f\left(v_{2}\right) \\
  \vdots \\
  f\left(v_{n}\right)
  \end{array}\right]
$$
For example, $f$ can be temperature, voltage, coordinate or $0-1$ indicator of $S \subseteq V$.
\par Remark that we can add or scalar multiply this function.
$$\left( f+g \right)\left(x\right) = f\left(x\right) + g\left(x\right),$$
$$c \cdot f \left(x\right) = f \left(c \cdot x\right). $$
So, $\{f: V \rightarrow \mathbb{R}\}$ is a vector space with dimension $n = \left| V \right|.$

\section{Key to SGT: The Quadratic Form}

\begin{definition}
  The quadratic form is defined to be
  $$\mathcal{E}[f]:=\frac{1}{2} \underset{u \sim v}{\mathbb{E}}\left[(f(\boldsymbol{u})-f(\boldsymbol{v}))^{2}\right]$$
  Where $u \sim v$ denotes we choose a uniform random edge $(u,v) \in E$.
\end{definition}

From the definition, we have some facts about the quadratic form.
\begin{itemize}
  \item $\mathcal{E}[f] \geq 0.$
  \item $\mathcal{E}[c \cdot f] = c^2 \cdot \mathcal{E}[f].$
  \item $\mathcal{E}[f + c] = \mathcal{E}[f].$
\end{itemize}
Intuitively, The quadratic form is small if and only if $f$'s value don't vary much along edges.
\par For example, if we take $S \subseteq V$ and $f = 1_{S}$ (the indicator function):
$$
f(v)=\left\{
  \begin{array}{ll}
  1, & \text { if } v \in S \\
  0, & \text { if } v \notin S
  \end{array}
  \right.
$$
Then we have:

\begin{align*}
  \mathcal{E}[f] &= \frac{1}{2} \cdot \underset{u \sim v}{\mathbb{E}}[(1_{s}(u) - 1_{s}(v))^2] \\
  &= \frac{1}{2} \cdot \underset{u \sim v}{\mathbb{E}}[1_{\{(u,v) \text{ cross the cut } (S, \bar{S})\}}] \\
  &= \frac{1}{2} \cdot \{\text{fraction of edges on } \partial S\} \\
  &= \underset{u \sim v}{\text{Pr}} [ u \rightarrow v \text{ is stepping out of } S]
\end{align*}

\section{Standard Random Walk}
Next, we define a distribution over $V$. To choose a random vertex
\begin{itemize}
  \item choose a uniform random edge $(u, v)$ (direct).
  \item output $u$.
\end{itemize}
We denote this distribution by $\pi$.
\begin{fact}
  $\pi[u]$ is proportional to deg(u) and 
  $$\pi(u) = \frac{\text{deg}(u)}{2|E|}.$$
\end{fact}
If $G$ is regular, $\pi$ is a uniform distribution on $V$.
\begin{fact}
  If we draw a vertice $u \sim \pi$, and let $v$ be a uniform random neighbor of $u$, then the distribution of $(u, v)$ is identical to draw a uniform random edge $(u^{prime}, v^{prime})$, and $v$ is also distributed according to $\pi$.
\end{fact}
\begin{proof}
  The probability of choosing a vertice $u$ is $\pi(u) = \frac{\text{deg}(u)}{2\left|E\right|}$. 
  \begin{align*}
    \text{Pr}\left[\text{pick the edge }(u,v)\right] &= \text{Pr}\left[\text{pick a random neighbor } v \text{ of } u \text{|pick a random vertice } u \right] \cdot \pi(u)\\
    &= \frac{1}{\text{deg}(u)} \cdot \frac{\text{deg}(u)}{2|E|}\\
    &= \frac{1}{2|E|}
  \end{align*}
  
\end{proof}
%%%%%%%%%%%%%%%%%%%%%%%%%%%%%%%%%%%%%%%%%%%%%%%%%%%%%%%%%%%%%%%%%%%%%%%%%%%%%%%
\begin{corollary}
  Let $t \in \mathbb{N}$, draw a vertice $u \sim \pi$ and do standard random walk for $t$ steps, then $v$ is also distributed according to $\pi$.
\end{corollary}
We define $\pi$ as a invariant distribution. But now we have a new question, say $u_{0}$ is not distributed according to $\pi$, do a $t$ steps random walk from $u_{0}$, as $t \rightarrow \infty$, does the distribution of $v \rightarrow \pi$? The answer is not if $G$ is disconnected or $G$ is bipartite. Otherwise, the answer is \textbf{YES}.
\par but how fast does it converge? We should consider the spectral of the graph. For example, let $S \subseteq V$, if the $cut(S, \bar{S})$ is tiny, then the convergence time may be very large, but we have showed in this case $\mathcal{E}[1_{S}]$ is small. Intuitively, converge fastly if and only if $\mathcal{E}[f]$ never small.

\section{Enter Linear Algebra}
Let $f:V \rightarrow \mathbb{R}$ and $u \sim \pi$, then $f(u)$ is a real random variable. We can define the mean and variance of this random variable.
\begin{itemize}
  \item Mean: $\mathbb{E}[f] := \underset{u \sim \pi}{\mathbb{E}} [f(u)]$. For example, if $S \subseteq V$, and $f = 1_s$, then $\mathbb{E}[f] = \underset{u \sim \pi}{\mathbb{E}} [1_s(u)] = \underset{u \sim \pi}{Pr}[u \in S]$. This is the "weight"/"volumn" of $S$.
  \item Variance: $Var[f(u)] = \underset{u \sim \pi}{\mathbb{E}} [(f(u)-\mu)^2] = \underset{u \sim \pi}{\mathbb{E}} [f(u)^2] - \underset{u \sim \pi}{\mathbb{E}} [f(u)]^2 = \frac{1}{2} \underset{u \sim \pi, v \sim \pi}{\mathbb{E}} [(f(u)-f(v))^2]$. Recall that $\mathcal{E}[f]:=\frac{1}{2} \underset{u \sim v}{\mathbb{E}}\left[(f(\boldsymbol{u})-f(\boldsymbol{v}))^{2}\right]$.
\end{itemize}
We can define inner product between two functions.
\begin{definition}
  Let $f, g: V \rightarrow \mathbb{R}$, the inner product of $f, g$ is defines as $<f,g>_\pi := \underset{u \sim \pi}{\mathbb{E}[f(u)g(u)]}$
\end{definition}
Remark this is a vector space inner product.
\begin{itemize}
  \item $\left< f,g \right>_\pi = \left<g,f \right>_\pi$.
  \item $\left< c \cdot f+g,h \right>_\pi = c \cdot \left< f,h \right>_\pi + \left<g,h \right>_\pi$.
  \item $||f||_{2}^{2} = \left< f,f \right>_\pi = \underset{u \sim \pi}{\mathbb{E}}[f(u)^2] \geq 0$ and the equality holds if and only if $f \equiv 0$.
\end{itemize}
Let $S \subseteq V$, $f = 1_s$, then $||f||_1 := \underset{u \sim \pi}{\mathbb{E}}[|f(u)|] = \underset{u \sim \pi}{\mathbb{E}}[1_s(u)] = \underset{u \sim \pi}{Pr}[u \in S]$ = "volumn" of S. Notice that, $||f||_2^2 = \left<f,f \right> = \underset{u \sim \pi}{\mathbb{E}}[f(u)^2] = \underset{u \sim \pi}{\mathbb{E}}[|f(u)|] = ||f||_1$.

\section{Minimizing and Maxmizing the Quardratic Form}
Now we have a question, how small can $\mathcal{E}[f]$ be? The answer is $0$ if we take $f \equiv 0$. Is there a nontrival $f$ with $\mathcal{E}[f] = 0$? Again, recall the definition of the quadratic form $\mathcal{E}[f]:=\frac{1}{2} \underset{u \sim v}{\mathbb{E}}\left[(f(\boldsymbol{u})-f(\boldsymbol{v}))^{2}\right]$.

\begin{proposition}
  $\mathcal{E}[f] = 0$ if and only if $f$ is constant on each connected component of $G$ and the number of connected component of $G$ equals to the number of linear independent $f$ with $\mathcal{E}[f] = 0$.
\end{proposition}
If the components are $S_1, ..., S_l$, $1_{S_1}, 1_{S_2},...,1_{S_l}$ are linear independent, the subspace $\{f: \mathcal{E}[f]=0\} = \sum_{i=1}^{l} c_i \cdot 1_{S_i}$.
\par Next, we maxmize $\mathcal{E}[f]$. But recall we have $\mathcal{E}[c \cdot f] = c^2 \cdot \mathcal{E}[f]$, so our task is maxmizing $\mathcal{E}[f]$ subject to $Var[f] = 1 (\leq 1)$. Remark this is identical to maxmize $\mathcal{E}[f]$ subject to $||f||_2^2 = \mathbb{E}[f^2] = 1 (\leq 1)$, since $\mathbb{E}[f^2] = Ver[f] + \mathbb{E}[f]^2$.
\par Intuitively, the edge endpoints' value should be as far as possible. For what kind of $G$ will you be most successful? The answer is bipartite graph. If $G$ is bipartite, $V=(V_1, V_2)$, let $f=1_{V_1} - 1_{V_2}$:
$$
f(u)=\left\{
  \begin{array}{ll}
  +1, & \text { if } u \in V_1 \\
  -1, & \text { if } u \in V_2
  \end{array}
  \right.
$$
Now, we have
\begin{itemize}
  \item $\underset{u \sim \pi}{\mathbb{E}}[f(u)^2] = \mathbb{E}[1] = 1$.
  \item $\mathcal{E}[f] =\frac{1}{2} \underset{u \sim v}{\mathbb{E}}\left[(f(\boldsymbol{u})-f(\boldsymbol{v}))^{2}\right] = 2$.
\end{itemize}
\begin{proposition}
  $\mathcal{E}[f] \leq 2 ||f||_2^2 = 2 \mathbb{E}[f^2]$.
\end{proposition}
\begin{proof}
  \begin{align*}
    \mathcal{E}[f] =\frac{1}{2} \underset{u \sim v}{\mathbb{E}}\left[(f(\boldsymbol{u})-f(\boldsymbol{v}))^{2}\right] &= \frac{1}{2} \underset{u \sim \pi}{\mathbb{E}}[f(u)^2] + \frac{1}{2} \underset{v \sim \pi}{\mathbb{E}}[f(v)^2] - \underset{u \sim v}{E} [f(u) \cdot f(v)]\\
    & \leq \mathbb{E}[f^2] + \sqrt{\underset{u \sim v}{\mathbb{E}}[f(u)^2]} \cdot \sqrt{\underset{u \sim v}{\mathbb{E}}[f(v)^2]}\\
    &= 2 \mathbb{E}[f^2]
  \end{align*}
\end{proof}
\par \textbf{Exercise:} The equality $\mathcal{E}[f] = 2 \mathbb{E}[f^2]$ is possible if and only if $G$ is bipartite.
% bibliography goes here
\bibliographystyle{amsalpha}

\end{document}
%%%%%%%%%%%%%%%%%%%%%%%%%%%%%%%%%%%%%%%%%%%%%%%%%%%%%%%%%%%%%%%%%%%%%%%%%%%%%%%