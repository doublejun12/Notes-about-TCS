\documentclass{article}
\usepackage[T1]{fontenc}
\usepackage{hyperref, amssymb, amsmath, graphicx, subfigure}

\setlength{\oddsidemargin}{.25in}
\setlength{\evensidemargin}{.25in}
\setlength{\textwidth}{6in}
\setlength{\topmargin}{-0.4in}
\setlength{\textheight}{8.5in}

\setlength{\parindent}{0in}
\setlength{\parskip}{8pt}

\newcommand{\heading}[6]{
  \renewcommand{\thepage}{#1-\arabic{page}}
  \noindent
  \begin{center}
  \framebox{
    \vbox{
      \hbox to 5.78in { \textbf{#2} \hfill #3 }
      \vspace{4mm}
      \hbox to 5.78in { {\Large \hfill #6  \hfill} }
      \vspace{2mm}
      \hbox to 5.78in { \textit{Instructor: #4 \hfill #5} }
    }
  }
  \end{center}
  \vspace*{4mm}
}

\newtheorem{theorem}{Theorem}
\newtheorem{definition}[theorem]{Definition}
\newtheorem{remark}[theorem]{Remark}
\newtheorem{lemma}[theorem]{Lemma}
\newtheorem{corollary}[theorem]{Corollary}
\newtheorem{proposition}[theorem]{Proposition}
\newtheorem{claim}[theorem]{Claim}
\newtheorem{observation}[theorem]{Observation}
\newtheorem{fact}[theorem]{Fact}
\newtheorem{assumption}[theorem]{Assumption}

\newenvironment{proof}{\noindent{\bf Proof:} \hspace*{1mm}}{
	\hspace*{\fill} $\Box$ }
\newenvironment{proof_of}[1]{\noindent {\bf Proof of #1:}
	\hspace*{1mm}}{\hspace*{\fill} $\Box$ }
\newenvironment{proof_claim}{\begin{quotation} \noindent}{
	\hspace*{\fill} $\diamond$ \end{quotation}}
	
\newcommand{\lecture}[4]{\heading{#1}{Spectral Graph Theory}{#2}{Ryan O'Donnel}{Scribe: #4}{#3}}



%%%%%%%%%%%%%%%%%%%%%%%%%%%%%%%%%%%%%%%%%%%%%%%%%%%%%%%%%%%%%%%%%%%%%%%%%%%%%%%
% PLEASE MODIFY THESE FIELDS AS APPROPRIATE:
\newcommand{\lecturenum}{1} % lecture number
\newcommand{\lecturedate}{\today} % date of lecture (e.g., 'March 20, 2010')
\newcommand{\lecturetitle}{The Quadratic Form and Standard Random Walk} % lecture title
\newcommand{\scribename}{Shuangjun Zhang} % full name of scribe
% PUT HERE ANY PACKAGES, MACROS, etc., ADDED BY YOU
%
\usepackage{graphicx}
\usepackage{subfigure}
%%%%%%%%%%%%%%%%%%%%%%%%%%%%%%%%%%%%%%%%%%%%%%%%%%%%%%%%%%%%%%%%%%%%%%%%%%%%%%%


%%%%%%%%%%%%%%%%%%%%%%%%%%%%%%%%%%%%%%%%%%%%%%%%%%%%%%%%%%%%%%%%%%%%%%%%%%%%%%%
\begin{document}
\lecture{\lecturenum}{\lecturedate}{\lecturetitle}{\scribename}

\section{The Labelling Function}

We consider the undirected graph $G=\left(V, E\right) $ and it satisfies the following properties:
\begin{itemize}
  \item The graph is finite.
  \item Multiple parallel edges and self-loops are allowed.
  \item vertices of degree $0$ are not allowed.
\end{itemize}
For simplicity, maybe we assume $G$ is regular.

\par We can label the vertice set $V$ by real numbers: 
$$f: V \rightarrow \mathbb{R} \equiv\left[\begin{array}{c}
  f\left(v_{1}\right) \\
  f\left(v_{2}\right) \\
  \vdots \\
  f\left(v_{n}\right)
  \end{array}\right]
$$
For example, $f$ can be temperature, voltage, coordinate or $0-1$ indicator of $S \subseteq V$.
\par Remark that we can add or scalar multiply this function.
$$\left( f+g \right)\left(x\right) = f\left(x\right) + g\left(x\right),$$
$$c \cdot f \left(x\right) = f \left(c \cdot x\right). $$
So, $\{f: V \rightarrow \mathbb{R}\}$ is a vector space with dimension $n = \left| V \right|.$

\section{Key to SGT: The Quadratic Form}

\begin{definition}
  The quadratic form is defined to be
  $$\mathcal{E}[f]:=\frac{1}{2} \underset{u \sim v}{\mathbb{E}}\left[(f(\boldsymbol{u})-f(\boldsymbol{v}))^{2}\right]$$
  Where $u \sim v$ denotes we choose a uniform random edge $(u,v) \in E$.
\end{definition}

From the definition, we have some facts about the quadratic form.
\begin{itemize}
  \item $\mathcal{E}[f] \geq 0.$
  \item $\mathcal{E}[c \cdot f] = c^2 \cdot \mathcal{E}[f].$
  \item $\mathcal{E}[f + c] = \mathcal{E}[f].$
\end{itemize}
Intuitively, The quadratic form is small if and only if $f$'s value don't vary much along edges.
\par For example, if we take $S \subseteq V$ and $f = 1_{S}$ (the indicator function):
$$
f(v)=\left\{
  \begin{array}{ll}
  1, & \text { if } v \in S \\
  0, & \text { if } v \notin S
  \end{array}
  \right.
$$
Then we have:

\begin{align*}
  \mathcal{E}[f] &= \frac{1}{2} \cdot \underset{u \sim v}{\mathbb{E}}[(1_{s}(u) - 1_{s}(v))^2] \\
  &= \frac{1}{2} \cdot \underset{u \sim v}{\mathbb{E}}[1_{\{(u,v) \text{ cross the cut } (S, \bar{S})\}}] \\
  &= \frac{1}{2} \cdot \{\text{fraction of edges on } \partial S\} \\
  &= \underset{u \sim v}{\text{Pr}} [ u \rightarrow v \text{ is stepping out of } S]
\end{align*}

\section{Standard Random Walk}
Next, we define a distribution over $V$. To choose a random vertex
\begin{itemize}
  \item choose a uniform random edge $(u, v)$ (direct).
  \item output $u$.
\end{itemize}
We denote this distribution by $\pi$.
\begin{fact}
  $\pi[u]$ is proportional to deg(u) and 
  $$\pi(u) = \frac{\text{deg}(u)}{2|E|}.$$
\end{fact}
If $G$ is regular, $\pi$ is a uniform distribution on $V$.
%%%%%%%%%%%%%%%%%%%%%%%%%%%%%%%%%%%%%%%%%%%%%%%%%%%%%%%%%%%%%%%%%%%%%%%%%%%%%%%

% bibliography goes here
\bibliographystyle{amsalpha}

\end{document}
%%%%%%%%%%%%%%%%%%%%%%%%%%%%%%%%%%%%%%%%%%%%%%%%%%%%%%%%%%%%%%%%%%%%%%%%%%%%%%%