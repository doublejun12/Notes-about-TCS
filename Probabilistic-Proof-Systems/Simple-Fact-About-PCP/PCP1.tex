\documentclass{article}
\usepackage[T1]{fontenc}
\usepackage{hyperref, amssymb, amsmath, graphicx, subfigure}

\setlength{\oddsidemargin}{.25in}
\setlength{\evensidemargin}{.25in}
\setlength{\textwidth}{6in}
\setlength{\topmargin}{-0.4in}
\setlength{\textheight}{8.5in}

\setlength{\parindent}{0in}
\setlength{\parskip}{8pt}

\newcommand{\heading}[6]{
  \renewcommand{\thepage}{#1-\arabic{page}}
  \noindent
  \begin{center}
  \framebox{
    \vbox{
      \hbox to 5.78in { \textbf{#2} \hfill #3 }
      \vspace{4mm}
      \hbox to 5.78in { {\Large \hfill #6  \hfill} }
      \vspace{2mm}
      \hbox to 5.78in { \textit{Instructor: #4 \hfill #5} }
    }
  }
  \end{center}
  \vspace*{4mm}
}

\newtheorem{theorem}{Theorem}
\newtheorem{definition}[theorem]{Definition}
\newtheorem{remark}[theorem]{Remark}
\newtheorem{lemma}[theorem]{Lemma}
\newtheorem{corollary}[theorem]{Corollary}
\newtheorem{proposition}[theorem]{Proposition}
\newtheorem{claim}[theorem]{Claim}
\newtheorem{observation}[theorem]{Observation}
\newtheorem{fact}[theorem]{Fact}
\newtheorem{assumption}[theorem]{Assumption}

\newenvironment{proof}{\noindent{\bf Proof:} \hspace*{1mm}}{
	\hspace*{\fill} $\Box$ }
\newenvironment{proof_of}[1]{\noindent {\bf Proof of #1:}
	\hspace*{1mm}}{\hspace*{\fill} $\Box$ }
\newenvironment{proof_claim}{\begin{quotation} \noindent}{
	\hspace*{\fill} $\diamond$ \end{quotation}}
	
\newcommand{\lecture}[4]{\heading{#1}{Probabilistic Proof Systems}{#2}{Alessandro Chiesa}{Scribe: #4}{#3}}



%%%%%%%%%%%%%%%%%%%%%%%%%%%%%%%%%%%%%%%%%%%%%%%%%%%%%%%%%%%%%%%%%%%%%%%%%%%%%%%
% PLEASE MODIFY THESE FIELDS AS APPROPRIATE:
\newcommand{\lecturenum}{PCP1} % lecture number
\newcommand{\lecturedate}{\today} % date of lecture (e.g., 'March 20, 2010')
\newcommand{\lecturetitle}{Some Basic Facts about PCPs} % lecture title
\newcommand{\scribename}{Shuangjun Zhang} % full name of scribe
% PUT HERE ANY PACKAGES, MACROS, etc., ADDED BY YOU
%
\usepackage{graphicx}
\usepackage{subfigure}
%%%%%%%%%%%%%%%%%%%%%%%%%%%%%%%%%%%%%%%%%%%%%%%%%%%%%%%%%%%%%%%%%%%%%%%%%%%%%%%


%%%%%%%%%%%%%%%%%%%%%%%%%%%%%%%%%%%%%%%%%%%%%%%%%%%%%%%%%%%%%%%%%%%%%%%%%%%%%%%
\begin{document}
\lecture{\lecturenum}{\lecturedate}{\lecturetitle}{\scribename}

\section{Some Special Case}
\par We wish to understand $\mathcal{PCP}[\epsilon_{c}, \epsilon_{s}, \Sigma, l, q, r, ...]$ in different regimes. Let's start with some special cases to warm up.
\par Suppose there is no proof ($q = 0$):
\begin{itemize}
  \item $\mathcal{PCP}[q=0, r=0] = \mathcal{P}$.
  \item $\mathcal{PCP}[q=0, r=\mathcal{O}(\log n)] = \mathcal{P}$.
  \item $\mathcal{PCP}[q=0, r=\text{poly} (n)] = \mathcal{BPP}$.
\end{itemize}

\par Suppose there is no randomness ($r = 0$):
\begin{itemize}
  \item $\mathcal{PCP}[q=\text{poly} (n), r=0] = \mathcal{NP}$.
\end{itemize}

\par We denote by $\mathcal{PCP}$ the complexity class with no restrictions beyond "V is PPT". This means that $q = \text{poly}(n), r=\text{poly}(n)$ and allow for $l = \text{exp}(n), \left|\Sigma\right| = \text{exp}(n)$.

\section{Upper Bound and Lower Bound on PCPs}
\begin{theorem}[Upper Bound]
  $\mathcal{PCP} \subseteq \mathcal{NEXP}$. 
\end{theorem}
\begin{lemma}
  The proof length $l \leq 2^r q$ for non-adaptive verifiers, and $l \leq 2^r \left|\Sigma\right|^q q$ for adaptive verifiers. (in constructions $l$ is usually smaller than these upper bounds)
\end{lemma}
\begin{proof}
  For non-adaptive verifier, there are at most $2^r$ different query sets, and for adaptive one each answer from the proof can lead to a different next query.
\end{proof}

\begin{lemma}
  $\mathcal{PCP}[l, r] \subseteq \mathcal{NTIME}((2^r+l) \cdot \text{poly}(n))$.
\end{lemma}
\begin{proof}
  Suppose $(P,V)$ is a $\mathcal{PCP}$ system for $L$ where the $PCP$ verifier users $r$ random bits to query a proof of length $l$. Consider the decider:
  \begin{itemize}
    \item $D(x, \pi) :=$ For every $\rho \in \{0,1\}^r$ compute $b_{\rho} := V^{\pi} (x; \rho)$ and output $1$ if and only if $\Sigma_{\rho} b_{\rho}/2^{r} \geq 1 - \epsilon_{c}$
  \end{itemize}
  If $x \in  L$, then $\exists \pi$ s.t. $D(x, \pi) = 1$. If $x \notin L$ then $\forall \pi$, $D(x, \pi) = 0$.
\end{proof}
\par The upper bound theorem follows from this two lemma.
\begin{theorem}[Lower Bound]
  $\mathcal{PSPACE} \subseteq \mathcal{PCP}$
\end{theorem}
\begin{proof}
  We prove $\mathcal{IP} \subseteq \mathcal{PCP}$.
  \par Suppose that $(P, V)$ is a public-coin IP for L. Consider proofs in this format: $\pi = \{a_{r_1}\}_{r_1} \cup \{a_{r_1, r_2}\}_{r_1, r_2} \cup \{a_{r_1,...,r_k}\}_{r_1,...,r_k}$ The PCP verifier samples $r_1,...,r_k$ and accepts if the IP verifier accepts:
  $$
  V(x, a_{r_1}, a_{r_1, r_2},...,a_{r_1,...,r_k}; r_1,...,r_k) \overset{?}{=} 1.
  $$
  \begin{itemize}
    \item Completeness: consider the honest proof 
    $$\pi = \{P(x,r_1)\}_{r_1} \cup \{P(x,r_1,r_2)\}_{r_1, r_2} \cup \{P(x,r_1,...,r_k)\}_{r_1,...,r_k}.$$
    \item Soundness: any proof in the above format corresponds to an "unrolled" IP prover.
  \end{itemize}
\end{proof}
\par In summarize, $\mathcal{PSPACE} \subseteq \mathcal{PCP} \subseteq \mathcal{NEXP}$. We will see that $\mathcal{PCP} = \mathcal{NEXP}$ by recycling techniques (arithmetization, sumcheck) and using new ones (Low Degree Testing), we will also see how to "scale down" to get PCPs for $\mathcal{NP}$.
%%%%%%%%%%%%%%%%%%%%%%%%%%%%%%%%%%%%%%%%%%%%%%%%%%%%%%%%%%%%%%%%%%%%%%%%%%%%%%%

% bibliography goes here
\bibliographystyle{amsalpha}

\end{document}
%%%%%%%%%%%%%%%%%%%%%%%%%%%%%%%%%%%%%%%%%%%%%%%%%%%%%%%%%%%%%%%%%%%%%%%%%%%%%%%